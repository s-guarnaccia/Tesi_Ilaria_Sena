\begin{abstract}
    Le vulnerabilità presenti all'interno di applicazioni possono costituire un rischio significativo, suscettibile di attacchi da parte di diverse entità, i quali possono compromettere dati sensibili, interrompere servizi critici e causare danni notevoli. Pertanto, è cruciale che ogni entità operante nel campo dell'informatica, comprese le aziende non specializzate in sicurezza informatica, adotti le misure preventive appropriate.

    Blue Reply, un'azienda consulenziale focalizzata sullo sviluppo di applicazioni web, è un'azienda che si impegna nell'implementare pratiche di programmazione sicura al fine di garantire una maggiore sicurezza nei loro prodotti e soddisfare le esigenze dei propri clienti.
    
    La mia ricerca condotta presso Blue Reply sarà di natura prevalentemente teorica e si articolerà 
    in quattro diverse fasi. Inizialmente, verranno esposte le motivazioni che rendono cruciale l'analisi 
    delle vulnerabilità in ambito web a partire dall'innovazione digitale avvenuta nel 2020 a causa della pandemia,
    evidenziando l'urgenza del problema attuale tramite diversi grafici e dati numerici. 

    Successivamente, l'attenzione sarà rivolta alle principali minacce note che possono riscontrarsi su applicazioni web, prenderò in considerazione 
    le classificazioni e gli studi più recenti, analizzando le minacce ed alcune delle loro possibili risoluzioni; verranno inoltre anche indicati alcuni degli attacchi più recenti tramite le minace prese in analisi.

    Dopo di che si analizzeranno nel dettaglio le metodologie di valutazione delle 
    vulnerabilità e le più importanti best practice da adottare per garantire una programmazione sicura. In questa fase verranno illustrate le tecnologie e le applicazioni più utilizzate e verranno anche inclusi esempi veri e propri di valutazione del rischio in azienda per le applicazioni web.

    Come ultima fase del mio studio ho scelto di trattare le vulnerabilità delle applicazioni Cloud, argomento molto attuale con il quale l'azienda Blue Reply si sta iniziando ad interfacciare; verranno illustrate le vulnerabilità e le best practice più comuni.

    L'obiettivo di questa tesi è quindi quello apprendere i rischi legati ad una programmazione poco attenta sapendo riconoscere e correggere le vulnerabilità più note, 
     inoltre si vuole fornire un'analisi delle procedure fondamentali per identificare e valutare le vulnerabilità presenti all'interno di applicazioni web e cloud, offrendo un contributo significativo al campo della sicurezza informatica dell'azienda.

\end{abstract}